\chapter*{Заключение}
\addcontentsline{toc}{chapter}{Заключение}

В результате исследования было определено, что время алгоритмов Левенштейна и Дамерау-Левенштейна растет в геометрической прогрессии при увеличении длин строк. При выборе между ними стоит отдавать предпочтение алгоритму Левенштейна, так как он в 2 раза быстрее на длине строк выше 5 символов. Но лучшие показатели по времени дает матричная реализация алгоритма Левенштейна и его рекурсивная реализация с кешем. использование которых приводит к 14-кратному превосходству по времени работы уже на длине строки в 4 символа за счет сохранения необходимых промежуточных вычислений. При этом матричные реализации занимают довольно много памяти при большой длине строк.


Цель, которая была поставлена в начале лабораторной работы была достигнута, а также в ходе выполнения лабораторной работы были решены следующие задачи:

\begin{itemize}
	\item были изучены и реализованы алгоритмы нахождения расстояния Левенштейна и Дамерау-Левенштейна;
	\item были также изучены матричная реализация, а также реализация с использованием кеша в виде матрицы для алгоритма Левенштейна;
    \item проведен сравнительный анализ алгоритмов Левенштейна и Дамерау-Левенштейна, а также сравнение рекурсивной и матричной реализаций, матричной и с кешом реализаций алгоритма Левенштейна
	\item подготовлен отчет о лабораторной работе.
\end{itemize}