\chapter{Конструкторская часть}
В этом разделе будут представлены требования к вводу и программе, а также схемы алгоритмов вычисления расстояния Левенштейна и Дамерау-Левенштейна.

\section{Требования к вводу}
\begin{enumerate}
    \item На вход подаются две строки.
    \item Буквы верхнего и нижнего регистров считаются различными.
\end{enumerate}


\section{Требования к программе}
\begin{enumerate}
    \item На вход подается две строки - корректный случай, программа должна верно обрабатывать такую ситуацию.
    \item В качестве результата работы программа должна вывести число, которое будет являться расстояние Левенштейна (Дамерау-Левенштейна), а также при необходимости матрицу.
\end{enumerate}


\section{Схемы алгоритмов}
На рисунках \ref{img:lev}, \ref{img:lev_mat}, \ref{img:lev_cash} и \ref{img:dam_lev} представлены схемы алгоритмов вычисления расстояния Левенштейна и Дамерау-Левенштейна.

\img{0.5}{lev}{Схема рекурсивного алгоритма нахождения расстояния Левенштейна}
\img{0.5}{lev_mat}{Схема матричного алгоритма нахождения расстояния Левенштейна}
\img{0.5}{lev_cash}{Схема рекурсивного алгоритма нахождения расстояния Левенштейна с использованием кеша (матрицы)}
\img{0.35}{dam_lev}{Схема рекурсивного алгоритма нахождения расстояния Дамерау-Левенштейна}

\clearpage

\section*{Вывод}
В данном разделе были представлены требования к вводу и программе, а также схемы алгоритмов, рассматриваемых в лабораторной работе.