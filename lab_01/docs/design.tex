\chapter{Конструкторская часть}
В этом разделе будут представлено описание использумеых типов данных, а также схемы алгоритмов вычисления расстояния Левенштейна и Дамерау-Левенштейна.

\section{Описание используемых типов данных}
При реализации алгоритмов будут использованы следующие структуры данных:

\begin{itemize}
    \item две строки типа \textit{str};
    \item длина строки - целое число типа \textit{int};
    \item в матричной реализации алгоритма Левенштейна и рекурсивной реализации с кешем - матрица, которая является двумерным списком типа \textit{int}. 
\end{itemize}


\section{Сведения о модулях программы}
Программа состоит из двух модулей:
\begin{itemize}
	\item $main.py$ - файл, содержащий весь служебный код;
    \item $algorythms.py$ - файл, содержащий код всех алгоритмов. \newline
\end{itemize}


\section{Схемы алгоритмов}
На рисунках \ref{img:lev}-\ref{img:dam_lev} представлены схемы алгоритмов вычисления расстояния Левенштейна и Дамерау-Левенштейна.

\imgScale{0.5}{lev}{Схема рекурсивного алгоритма нахождения расстояния Левенштейна}
\imgScale{0.5}{lev_mat}{Схема матричного алгоритма нахождения расстояния Левенштейна}
\imgScale{0.5}{lev_cash}{Схема рекурсивного алгоритма нахождения расстояния Левенштейна с использованием кеша (матрицы)}
\imgScale{0.45}{dam_lev}{Схема рекурсивного алгоритма нахождения расстояния Дамерау-Левенштейна}

\clearpage

\section{Использование памяти}

С точки зрения замеров и сравнения используемой памяти, алгоритмы Левенштейна и Дамерау-Левенштейна не отличаются друг от друга.
Тогда рассмотрим только рекурсивную и матричную реализации данных алгоритмов.

Пусть:
\begin{itemize}
    \item n - длина строки S1
    \item m - длина строки S2
\end{itemize}

Тогда затраты по памяти будут такими:
\begin{itemize}
    \item алгоритм нахождения расстояния Левенштейна (рекурсивный), где для каждого вызова:

    \begin{itemize}
        \item для S1, S2 - (n + m) * sizeof(char)
        \item для n, m - 2 * sizeof(int)
        \item доп. переменные - 2 * sizeof(int)
        \item адрес возврата
    \end{itemize}

    \item алгоритм нахождения расстояния Левенштейна с использованием кеша в виде матрицы (память на саму матрицу: ((n + 1) * (m + 1)) * sizeof(int)) (рекурсивный), где для каждого вызова:

    \begin{itemize}
        \item для S1, S2 - (n + m) * sizeof(char)
        \item для n, m - 2 * sizeof(int)
        \item доп. переменные - 2 * sizeof(int)
        \item ссылка на матрицу - 8 байт
        \item адрес возврата
    \end{itemize}

    \item алгоритм нахождения расстояния Дамерау-Левенштейна (рекурсивный), где для каждого вызова:

    \begin{itemize}
        \item для S1, S2 - (n + m) * sizeof(char)
        \item для n, m - 2 * sizeof(int)
        \item доп. переменные - 2 * sizeof(int)
        \item адрес возврата
    \end{itemize}


    \item алгоритм нахождения расстояния Левенштейна (матричный):

    \begin{itemize}
        \item для матрицы - ((n + 1) * (m + 1)) * sizeof(int)) 
        \item текущая строка матрицы - (n + 1) * sizeof(int)
        \item для S1, S2 - (n + m) * sizeof(char)
        \item для n, m - 2 * sizeof(int)
        \item доп. переменные - 3 * sizeof(int)
        \item адрес возврата
    \end{itemize}
\end{itemize}

\section{Вывод}
В данном разделе были представлено описание используемых типов данных, а также схемы алгоритмов, рассматриваемых в лабораторной работе.