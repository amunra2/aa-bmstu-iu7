\chapter*{Введение}
\addcontentsline{toc}{chapter}{Введение}

Операции работы со строками являются очень важной частью всего программирования. Часто возникает потребность в использовании строк для различных задач - обычные статьи, записи в базу данных и так далее. Отсюда возникает несколько важных задач, для решения которых нужны алгоритмы сравнения строк. Об этих алгоритмах и пойдет речь в данной работе. 
Подобные алгоритмы используются при:
\begin{itemize}
	\item исправлении ошибок в тексте, предлагая заменить введенное слово с ошибкой на наиболее подходящее;
    \item поиске слова в тексте по подстроке;
    \item сравнении целых текстовых файлов. \newline
\end{itemize}



\textbf{Цель работы:} изучение, реализация и исследование алгоритмов нахождения расстояний Левенштейна и Дамерау--Левенштейна. \newline

\textbf{Задачи работы.}
\begin{enumerate}
	\item Изучение алгоритмов Левенштейна и Дамерау--Левенштейна.
    \item Реализация алгоритмов нахождения расстояний Левенштейна и Дамерау--Левенштейна.
    \item Сравнение алгоритмов по затрачиваемым ресурсам -- времени и памяти.
    \item Приведение экспериментов, подтверждающих различия в эффективности алгоритмов по времени, используя разработанное программное обеспечение.
	\item Описание и обоснование полученных результатов в отчете о выполненной лабораторной работе, выполненного как расчётно-пояснительная записка к работе.
\end{enumerate}
