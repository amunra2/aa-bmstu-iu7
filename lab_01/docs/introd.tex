\chapter*{Введение}
\addcontentsline{toc}{chapter}{Введение}

Операции работы со строками являются очень важной частью всего программирования. Часто возникает потребность в использовании строк для различных задач - обычные статьи, записи в базу данных и так далее. Отсюда возникает несколько важных задач, для решения которых нужны алгоритмы сравнения строк. Об этих алгоритмах и пойдет речь в данной работе. 
Подобные алгоритмы используются при:
\begin{itemize}
	\item исправлении ошибок в тексте, предлагая заменить введенное слово с ошибкой на наиболее подходящее;
    \item поиске слова в тексте по подстроке;
    \item сравнении целых текстовых файлов. \newline
\end{itemize}



\textbf{Целью данной работы} является изучение, реализация и исследование алгоритмов нахождения расстояний Левенштейна и Дамерау--Левенштейна. 
Для достижения поставленной цели необходимо выполнить следующие задачи:
\begin{itemize}
	\item изучить и реализовать алгоритмы Левенштейна и Дамерау--Левенштейна;
    \item провести тестирование по времени и по памяти для алгоритмов Левенштейна и Дамерау-Левенштейна;
    \item провести сравнительный анализ по времени рекурсивной и матричной реализации алгоритма нахождения расстояния Левенштейна;
    \item провести сравнительный анализ по времени матричной и с кешем реализации алгоритма нахождения расстояния Левенштейна;
    \item провести сравнительный анализ по времени алгоритмов нахождения расстояния Левенштейна и Дамерау-Левенштейна;
	\item описать и обосновать полученные результаты в отчете о выполненной лабораторной работе, выполненного как расчётно-пояснительная записка к работе.
\end{itemize}
