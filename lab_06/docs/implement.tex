\chapter{Технологическая часть}

В данном разделе будут рассмотрены средства реализации, а также представлены листинги алгоритма полного перебора путей и муравьиного алгоритма.

\section{Средства реализации}
В данной работе для реализации был выбран язык программирования $Python \cite{python-lang}$. В текущей лабораторной работе требуется замерить процессорное время для выполняемой программы, а также построить графики. Все эти инструменты присутствуют в выбранном языке программирования.

Время работы было замерено с помощью функции \textit{process\_time(...)} из библиотеки $time \cite{python-lang-time}$.


\section{Листинги кода}

В листинге \ref{lst:full_comb} представлен алгоритм полного перебора путей, а в листингах \ref{lst:ant_alg}-\ref{lst:upd_pher} - муравьиный алгоритм и дополнительные к нему функции.

\clearpage

\begin{center}
    \captionsetup{justification=raggedright,singlelinecheck=off}
    \begin{lstlisting}[label=lst:full_comb,caption=Алгоритм полного перебора путей]
def full_combinations(matrix, size):
	places = np.arange(size)
	places_combs = list()

	for combination in it.permutations(places):
		comb_arr = list(combination)

		places_combs.append(comb_arr)

	min_dist = float("inf")

	for i in range(len(places_combs)):
		places_combs[i].append(places_combs[i][0])

		cur_dist = 0

		for j in range(size):
			start_city = places_combs[i][j]
			end_city = places_combs[i][j + 1]

			cur_dist += matrix[start_city][end_city]

		if (cur_dist < min_dist):
			min_dist = cur_dist

			best_way = places_combs[i]

	return min_dist, best_way
\end{lstlisting}
\end{center}

\clearpage

\begin{center}
    \captionsetup{justification=raggedright,singlelinecheck=off}
    \begin{lstlisting}[label=lst:ant_alg,caption=Муравьиный алгоритм]
def ant_algorythm(matrix, places, alpha, beta, k_evaporation, days):
	q = calc_q(matrix, places)

	best_way = []
	min_length = float("inf")

	pheromones = get_pheromones(places)
	visibility = get_visibility(matrix, places)

	ants = places

	for day in range(days):
		route = np.arange(places)

		visited = get_visited_places(route, ants)

		for ant in range(ants):
			while (len(visited[ant]) != ants):
				pk = find_posibilyties(pheromones, visibility, visited, places, ant, alpha, beta)  

				chosen_place = choose_next_place_by_posibility(pk)

				visited[ant].append(chosen_place - 1)
		
			visited[ant].append(visited[ant][0])

			cur_length = calc_length(matrix, visited[ant]) 

			if (cur_length < min_length):
				min_length = cur_length
				best_way = visited[ant]

		pheromones = update_pheromones(matrix, places, visited, pheromones, q, k_evaporation)

	return min_length, best_way
\end{lstlisting}
\end{center}


\clearpage


\begin{center}
    \captionsetup{justification=raggedright,singlelinecheck=off}
    \begin{lstlisting}[label=lst:prob_arr,caption=Алгоритм нахождения массива вероятностных переходов в непосещенные города]
def find_posibilyties(pheromones, visibility, visited, places, ant, alpha, beta):
	pk = [0] * places

	for place in range(places):
		if place not in visited[ant]:
			ant_place = visited[ant][-1]

			pk[place] = pow(pheromones[ant_place][place], alpha) * \
				pow(visibility[ant_place][place], beta)
		
		else:
			pk[place] = 0

	sum_pk = sum(pk)

	for place in range(places):
		pk[place] /= sum_pk  

	return pk
\end{lstlisting}
\end{center}

\clearpage

\begin{center}
    \captionsetup{justification=raggedright,singlelinecheck=off}
    \begin{lstlisting}[label=lst:choose_next,caption=Алгоритм нахождения следующего города на основании рандома]
def choose_next_place_by_posibility(pk):
	posibility = random()
	choice = 0
	chosen_place = 0

	while ((choice < posibility) and (chosen_place < len(pk))):
		choice += pk[chosen_place]
		chosen_place += 1

	return chosen_place
\end{lstlisting}
\end{center}


\begin{center}
    \captionsetup{justification=raggedright,singlelinecheck=off}
    \begin{lstlisting}[label=lst:upd_pher,caption=Алгоритм обновления матрицы феромонов]
def update_pheromones(matrix, places, visited, pheromones, q, k_evaporation):
	ants = places

	for i in range(places):
		for j in range(places):
			delta_pheromones = 0

			for ant in range(ants):
				length = calc_length(matrix, visited[ant])
				delta_pheromones += q / length

			pheromones[i][j] *= (1 - k_evaporation)
			pheromones[i][j] += delta_pheromones

			if (pheromones[i][j] < MIN_PHEROMONE):
				pheromones[i][j] = MIN_PHEROMONE

	return pheromones
\end{lstlisting}
\end{center}

\section{Сведения о модулях программы}
Программа состоит из двух модулей:
\begin{itemize}
	\item \textit{main.py} - файл, содержащий меню программы, а также весь служебный код;
    \item \textit{algorythms.py} - файл, содержащий реализацию алгоритма полного перебора и муравьиного алгоритма.
\end{itemize}


\section{Функциональные тесты}

В таблице \ref{tbl:functional_test} приведены тесты для функций программы. Тесты \textit{для всех функций} пройдены успешно.

\begin{center}
    \captionsetup{justification=raggedright,singlelinecheck=off}
    \begin{longtable}[c]{|c|c|c|c|c|}
    \caption{Функциональные тесты\label{tbl:functional_test}} \\ \hline
		Матрица смежности & Ожидаемый результат & Результат программы \\
		\hline
		$ \begin{pmatrix}
			0 &  4 &  2 &  1 & 7 \\
			4 &  0 &  3 &  7 & 2 \\
			2 &  3 &  0 & 10 & 3 \\
			1 &  7 & 10 &  0 & 9 \\
			7 &  2 &  3 &  9 & 0
		\end{pmatrix}$ &
		15, [0, 2, 4, 1, 3, 0] &
		15, [0, 2, 4, 1, 3, 0] \\

		$ \begin{pmatrix}
			0 & 1 & 2 \\
			1 & 0 & 1 \\
			2 & 1 & 0	
		\end{pmatrix}$ &
		4, [0, 1, 2, 0] &
		4, [0, 1, 2, 0] \\

		$ \begin{pmatrix}
			0 & 15 & 19 & 20 \\
			15 &  0 & 12 & 13 \\
			19 & 12 &  0 & 17 \\
			20 & 13 & 17 &  0
		\end{pmatrix}$ &
		64, [0, 1, 2, 3, 0] &
		64, [0, 1, 2, 3, 0] \\
		\hline
	\end{longtable}
\end{center}

\section{Вывод}

Были представлены листинги всех алгоритмов -- полного перебора и муравьиного. Также в данном разделе была приведена информации о выбранных средствах для разработки алгоритмов и сведения о модулях программы, проведено функциональное тестирование.
