\chapter{Аналитическая часть}
В этом разделе будет представлена информация о задаче коммивояжера, а также о путях ее решения -- полным перебором или муравьиным алгоритмом.


\section{Задача коммивояжера}

\textbf{Задача коммивояжера} \cite{kom-task} -- (задача о бродячем торговце) одна из самых важных задач всей транспортной логистики, в которой рассматриваются вершины графа, а также матрица смежности (для расстояния между вершинами). Суть -- найти такой порядок посещения вершин графа, при котором путь будет минимален, каждая вершина будет посещена лишь один раз, а возврат произойдет в начальную вершину.


\section{Алгоритм полного перебора для решения задачи коммивояжера}

\textbf{Полный перебор для задачи коммивояжера} \cite{kom-task-all} -- имеет высокую сложность алгоритма ($n!$). Суть в полном переборе всех возможных путей в графе и выбор наименьшего из них. Решение будет получено, но имеются большие затраты по времени выполнения при уже небольшом количестве вершин в графе.


\section{Муравьиный алгоритм}

\textbf{Муравьиный алгоритм} \cite{ant-alg} -- основан на принципе поведения колонии муравьев.

Муравьи действуют, ощущая некий феромон. Каждый муравей, чтобы другие могли ориентироваться, оставляет на своем пути феромоны. При большом количестве прохождения муравьев, наибольшее количество феромона остается на оптимальном пути.

Суть в том, что отдельно взятый муравей мало, что может, поскольку способен выполнять только максимально простые задачи. Но при условии большого количества таких муравьев они могу самоорганизовываться в большие очереди для решения сложных задач.

Пусть муравей имеет следующие характеристики:
\begin{itemize}
	\item зрение -- способен определить длину ребра;
	\item память -- запоминает пройденный маршрут;
	\item обоняние -- чувствует феромон.
\end{itemize}


Также введем целевую функцию \eqref{d_func}.

\begin{equation}
	\label{d_func}
	\eta_{ij} = 1 / D_{ij},
\end{equation}
где $D_{ij}$ — расстояние из текущего пункта $i$ до заданного пункта $j$.


А также понадобится формула вычисления вероятности перехода в заданную точку \eqref{posib}.

\begin{equation}
	\label{posib}
	P_{kij} = \begin{cases}
		\frac{\tau_{ij}^a\eta_{ij}^b}{\sum_{q=1}^m \tau^a_{iq}\eta^b_{iq}}, \textrm{вершина не была посещена ранее муравьем k,} \\
		0, \textrm{иначе}
	\end{cases}
\end{equation}
где $a$ -- параметр влияния длины пути, $b$ -- параметр влияния феромона, $\tau_{ij}$ -- расстояния от города $i$ до $j$, $\eta_{ij}$ -- количество феромонов на ребре $ij$.

После завершения движения всех муравьев, формула обновляется феромон по формуле \eqref{update_phero_1}:
\begin{equation}
	\label{update_phero_1}
		\tau_{ij}(t+1) = (1-p)\tau_{ij}(t) + \Delta \tau_{ij}.
\end{equation}
При этом
\begin{equation}
\label{update_phero_2}
 \Delta \tau_{ij} = \sum_{k=1}^N \tau^k_{ij},
\end{equation}
где
\begin{equation}
	\label{update_phero_3}
		 \Delta\tau^k_{ij} = \begin{cases}
		Q/L_{k}, \textrm{ребро посещено k-ым муравьем,} \\
		0, \textrm{иначе}
	\end{cases}
\end{equation}


Путь выбирается по следующей схеме:

\begin{enumerate}
	\item Каждый муравей имеет список запретов -- список уже посещенных городов (вершин графа).
	\item Муравьиной зрение -- отвечает за желание посетить вершину.
	\item Муравьиное обоняние -- отвечает за ощущение феромона на определенном пути (ребре). При этом количество феромона на пути (ребре) в момент времени $t$ обозначается как $\tau_{i, j} (t)$.
	\item После прохождения определенного ребра муравей откладывает на нем некотрое количество феромона, которое показывает оптимальность сделанного выбора (это кол-во вычисляется по формуле \eqref{update_phero_3})
\end{enumerate}



\section{Вывод}

В данном разделе была рассмотрена задача коммивояжера, а также полный перебор для ее решения и муравьиный алгоритм.

Программа будет получать на вход матрицу смежности, для которой можно будет выбрать один из алгоритмов поиска путей -- полным перебором или муравьиный. Также можно будет ввести коэффиценты и количество дней для работы муравьиного алгоритма. При неверном вводе какого-то из значений будет выдаваться сообщение об ошибке.

Реализуемое ПО дает возможность получить минимальную сумму пути, а также сам путь, используя один из алгоритмов. Также имеется возможность провести тестирование по времени для разных размеров матриц.
