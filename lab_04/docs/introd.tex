\chapter*{Введение}
\addcontentsline{toc}{chapter}{Введение}


В процессе развития компьютерных систем программисты стали все чаще сталкиваться с задачами, где трубется большое количество различных вычислений. Все вычисления происходят на одном процессоре. Как одно из решений, вычисления можно проводить на разных компьютерах. Но балгодаря развитию процессоров, было предложено распараллелить вычисления на одном процессоре, используя потоки, каждый из которых запусакется на отдельном логическом ядре. Тем самым появился термин "многопоточность", который заключается в разделении процессов, выполняя их параллельно. \newline


\textbf{Целью данной работы} является изучение параллельных вычислений на основе построения пучка отрезков по алгоритму Брезенхема. 
Для достижения поставленной цели необходимо выполнить следующие задачи:
\begin{itemize}
	\item изучить основы распараллеливания вычислений;
    \item реализовать алгоритм построения пучка отрезков по Брезенхему с использованием многопоточности и без;
    \item провести сравнительный анализ по времени на одной длине пучка и различном количестве потоков, а также без многопочности;
    \item провести сравнительный анализ по времени при разной длине пучка с использованием многопоточности и без;
	\item описать и обосновать полученные результаты в отчете о выполненной лабораторной работе, выполненного как расчётно-пояснительная записка к работе.
\end{itemize}
