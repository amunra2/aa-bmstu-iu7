\chapter*{Заключение}
\addcontentsline{toc}{chapter}{Заключение}

В результате исследования было определено, что на данном ноутбуке лучшие результаты достигаются на 4 потоках, так как имеется всего 4 потока из-за наличия всего 4 логических ядер, а также из-за особенности реализации алгоритма, что остаток, который не удалось распределить между всеми потоками, отдается последнему потоку (что не происходит при 4 потоках). Преимущество во времени работы при использовании многопоточности на 4 потоках над алгоритмом, который реализован без использования многопоточности, велико - примерно в 3.7 раза лучше. Также важно, что при увеличении количества потоков растет время выполнения программы из-за того, что также время уходит на поддержку каждого потока, поэтому при большом количестве потоков (около 64) их использованием станет нецелесообразным. При этом на 4 потоках при увеличении длины отрезка в спектре видно, что многопоточность на длине в 3000 лучше реализации без многопоточности в 3 раза, а при 10 тысячах -- в 3.3 раза, то есть при большой длине отрезка стоит использовать многпоточную реализацию.

Цель, которая была поставлена в начале лабораторной работы была достигнута, а также в ходе выполнения лабораторной работы были решены следующие задачи:

\begin{itemize}
	\item были изучены принципы многопотчности и реализованы алгоритмы построения спектров по алгоритму Брезенехема с использованием многопоточности и без нее;
    \item был реализован алгоритм Брезенехема для генерации отрезка;
	\item проведен сравнительный анализ по времени алгоритмов с многопоточностью на разном количестве потоков и без многопоточности;
	\item проведен сравнительный анализ по времени алгоритмов с многопоточностью и без нее на разной длине отрезка в спектре;
	\item подготовлен отчет о лабораторной работе.
\end{itemize}