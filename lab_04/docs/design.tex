\chapter{Конструкторская часть}
В этом разделе будут представлено описание используемых типов данных, а также схемы алгоритмов построения пучка отрезков по методу Брезенхема с распараллеливанием и без него.

\section{Описание используемых типов данных}

При реализации алгоритмов будут использованы следующие типы данных:

\begin{itemize}
	\item количество потоков - целое число типа \textit{int};
	\item длина отрезка - целое число типа \textit{int};
	\item структура \textit{request} - содержит информацию о типе алгоритма (с использованием многопоточности или без нее), необоходимые параметры о канвасе, а также параметр \textit{is\_draw} - определяет, нужно ли отрисовывать на канвас спектр отрезков Брезенхема (не отрисовывается при замере времени, так как происходит множество прогонов);
	\item структура \textit{beam\_settings} - содержит информацию о спектре - длина отрезка в спектре, количество потоков при построении (если требуется);
	\item структура \textit{point} - содержит информацию о точке - ее координаты по $x$ и $y$.
\end{itemize}


\section{Структура разрабатываемого ПО}

В данном ПО будет реализован метод структурного программирования. Для взаимодействия с пользователем будет разработан интерфейсом, на котором можно будет задать нужные параметры, выбрать - распараллеливать ли решаемую задачу или нет, а также интерфейс будет содержать окно вывода того, что отрисовалось (канвас). 

Для работы будут разработаны следующие процедуры:

\begin{itemize}
	\item процедура, реализующая построение отрезка по алгоритму Брезенехема;
	\item процедура построения спектров без использования многопоточности, входные данные - длина отрезка в спектре, выходные - множество отрезков на канвасе;
	\item процедура построения спектров c использованием многопоточности, входные данные - длина отрезка в спектре, количество потоков, выходные - множество отрезков на канвасе;
	\item процедуры замера времени построения спектров отрезков по Брезенехему с использованием многопоточности и без, входные данные - длина отрезка в спектре (если время замеряется для разного количества потоков), количество потоков (если замеряется время для разной длины отрезков в спектре), выходные - результаты замеров времени;
	\item процедура для построения графиков по полученным временным замерам, входные данные - замеры времени, выходные - график.
\end{itemize}


\section{Схемы алгоритмов}
На рисунке \ref{img:bres_alg} представлена схема алгоритма для построения отрезка по алгоритму Брезенхема. На рисунках \ref{img:no_par} схема алгоритма построения спектра отрезков по Брезенехему без использования многопоточности, а на \ref{img:par} - с использованием многопоточности. Также на рисунке \ref{img:part_par} представлена схема алгоритма построения части отрезков Брезенхема для параллельного вычисления.

\imgScale{0.4}{bres_alg}{Схема алгоритма Брезенхема для построения отрезка}
\imgScale{0.6}{no_par}{Схема построения спектра отрезков по Брезенехему (без многопоточности)}
\imgScale{0.6}{par}{Схема построения спектра отрезков по Брезенехему (с многопоточностью)}
\imgScale{0.6}{part_par}{Схема построения части отрезков спектра для текущего потока}

\clearpage


\section{Классы эквивалентности при тестировании}

Для тестирования выделены классы эквивалентности, представленные ниже.

\begin{enumerate}
	\item Не введена длина отрезка в спектре или длина меньше или равно нулю.
	\item Не введено количество потоков или их количество меньше или равно нулю.
	\item Корректный ввод всех параметров.
\end{enumerate}


\section{Вывод}

В данном разделе были построены схемы алгоритмов умножения матриц рассматриваемых в лабораторной работе, были описаны классы эквивалентности для тестирования, структура программы.
