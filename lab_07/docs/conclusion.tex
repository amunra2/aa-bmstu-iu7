\chapter*{Заключение}
\addcontentsline{toc}{chapter}{Заключение}

В результате исследования было определено, что самым быстрым алгоритмом поиска является метод разбиения на сегменты, которые примерно в 10 раз быстрее бинарного поиска и в сотни раз быстрее алгоритма поиска полным перебором, время которого зависит от удаленности ключа в словаре. Также бинарный алгоритм является алгоритмом, которому для нахождения элемента в словаре нужно меньше всего сравнений. Так, в словаре, который используется в данной лабораторной работе, для поиска информации по ключу требуется максимум 12 сравнений (274 -- для алгоритма разбиения на сегменты, 2695 сравнений -- для алгоритма полного перебора).  

Цель, которая была поставлена в начале лабораторной работы была достигнута, а также в ходе выполнения лабораторной работы были решены следующие задачи:

\begin{itemize}
	\item были изучены алгоритмы поиска информации по ключу в словаре -- полным перебором, бинарного поиска и поиска с разбиением на сегменты;
    \item были реализованы алгоритмы полного перебора, бинарного поиска и поиска с разбиением на сегменты;
	\item проведен сравнительный анализ по времени алгоритмов на всех ключах словаря;
	\item проведена сравнительный анализ по количеству сравнений, необходимых алгоритму для нахождения каждого ключа в словаре;
	\item подготовлен отчёт о лабораторной работе.
\end{itemize}