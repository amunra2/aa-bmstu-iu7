\chapter{Конструкторская часть}
В этом разделе будут представлено описание используемых типов данных, а также схемы алгоритма полного перебора, бинарного поиска и поиска разбиением на сегменты.

\section{Описание используемых типов данных}

При реализации алгоритмов будут использованы следующие типы данных:

\begin{itemize}
	\item название файла - строка типа \textit{str};
	\item список ключей или значений - список типа \textit{str};
	\item словарь для хранения информации о футболисте по его имени (ключу) - встроенный тип \textit{dict} в \textit{Python}.
\end{itemize}


\section{Структура разрабатываемого ПО}

В данном ПО будет реализован метод структурного программирования. Для взаимодействия с пользователем будет разработано меню, которое будет предоставлять возможность выбрать нужный алгоритм - полного перебора, бинарного поиска или поиска сегментами, а также построить графики для сравнения времени выполнения алгоритмов и гистограммы для анализа количества сравнений для каждого алгоритма. 

Для работы будут разработаны следующие процедуры:

\begin{itemize}
	\item процедура, реализующая создание словаря, используя информацию из файла, входные данные - имя файла, выходные - заполненный словарь;
	\item процедура вывода полученного словаря на экран (для отладки), входные данные - словарь, выходные - выведенный на экран словарь;
	\item процедура, реализующая алгоритм поиска полным перебором, входные данные - словарь значений, ключ (по которому будет производиться поиск информации), выходные - количество сравнений, а также найденное по ключу значение в словаре;
	\item процедура, реализующая алгоритм бинарного поиска, входные данные - словарь значений, ключ (по которому будет производиться поиск информации), выходные - количество сравнений, а также найденное по ключу значение в словаре;
	\item процедура, реализующая алгоритм поиска сегментами, входные данные - словарь значений, ключ (по которому будет производиться поиск информации), выходные - количество сравнений, а также найденное по ключу значение в словаре;
	\item процедуры замера времени алгоритмов полного перебора, бинарного поиска и поиска сегментам для всех ключей словаря, входные данные - словарь, выходные - результаты замеров времени;
	\item процедура для построения гистограмм для полученных значений сравнений для каждого ключа в словаре, входные данные - словарь, выходные - гистограммы (отсортированная (по убыванию количества сравнений) и нет); 
	\item процедура для построения графиков по полученным временным замерам, входные данные - замеры времени, выходные - график. 
\end{itemize}


\section{Схемы алгоритмов}
На рисунке \ref{img:full_search} представлена схема алгоритма поиска в словаре с помощью полного перебора, а на рисунках \ref{img:binary_search} и \ref{img:segment_search} схемы алгоритмов бинарного и сегментного поисков соответственно.

\imgScale{0.5}{full_search}{Схема алгоритма поиска полным перебором}
\imgScale{0.5}{binary_search}{Схема алгоритма бинарного поиска}
\imgScale{0.5}{segment_search}{Схема алгоритма поиска с разбиением словаря на сегменты}

\clearpage


\section{Классы эквивалентности при тестировании}

Для тестирования выделены классы эквивалентности, представленные ниже.

\begin{enumerate}
	\item Неверно выбран пункт меню - не число или число, меньшее 0 или большее 7.
	\item Неверно введен ключ - пустая строка.
	\item Введенного ключа нет в словаре.
	\item Введенный ключ есть в словаре.
\end{enumerate}


\section{Вывод}

В данном разделе были построены схемы алгоритмов, рассматриваемых в лабораторной работе, были описаны классы эквивалентности для тестирования, структура программы.
