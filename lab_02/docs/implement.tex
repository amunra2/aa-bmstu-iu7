\chapter{Технологическая часть}

В данном разделе будут рассмотрены средства реализации, а также представлены листинги алгоритмов определения расстояния Левенштейна и Дамерау-Левенштейна.

\section{Средства реализации}
В данной работе для реализации был выбран язык программирования $Python \cite{python-lang}$. В текущей лабораторной работе требуется замерить процессорное время для выполняемой программы, а также построить графики. Все эти инструменты присутствуют в выбранном языке программирования.

Время работы было замерено с помощью функции \textit{process\_time(...)} из библиотеки $time \cite{python-lang-time}$.


\section{Листинги кода}

В листингах \ref{lst:stand_alg}-\ref{lst:opt_vin_alg} представлены реализации алгоритмов умножения матриц - стандартного, Винограда и оптимизированного алгоритма Винограда.

\clearpage

\begin{center}
    \captionsetup{justification=raggedright,singlelinecheck=off}
    \begin{lstlisting}[label=lst:stand_alg,caption=Стандартный алгоритм умножения матриц]

\end{lstlisting}
\end{center}


\begin{center}
    \captionsetup{justification=raggedright,singlelinecheck=off}
    \begin{lstlisting}[label=lst:lev_mat,caption=Алгоритм нахождения расстояния Левенштейна (матричный)]

\end{lstlisting}
\end{center}


\begin{center}
    \captionsetup{justification=raggedright,singlelinecheck=off}
    \begin{lstlisting}[label=lst:lev_cach,caption=Алгоритм нахождения расстояния Левенштейна c использованием кеша в виде матрицы]

\end{lstlisting}
\end{center}


\begin{center}
    \captionsetup{justification=raggedright,singlelinecheck=off}
    \begin{lstlisting}[label=lst:dam_lev_rec,caption=Алгоритм нахождения расстояния Дамерау-Левенштейна (рекурсивный)]

\end{lstlisting}
\end{center}

\section{Функциональные тесты}

В таблице \ref{tbl:functional_test} приведены тесты для функций, реализующих алгоритмы нахождения расстояния Левенштейна и Дамерау-Левенштейна. Тесты \textit{для всех алгоритмов} пройдены успешно.

\begin{table}[h]
	\begin{center}
        \begin{threeparttable}
        \captionsetup{justification=raggedright,singlelinecheck=off}
		\caption{\label{tbl:functional_test} Функциональные тесты}
		\begin{tabular}{|c|c|c|c|c|}
			\hline
			& & & \multicolumn{2}{c|}{Ожидаемый результат} \\
			\hline
			№&Строка 1&Строка 2&Левенштейн&Дамерау-Л. \\
			\hline
            1&"пустая строка"&"пустая строка"&0&0 \\
            \hline
            2&"пустая строка"&слово&5&5 \\
            \hline
            3&проверка&"пустая строка"&8&8 \\
            \hline
            4&ремонт&емонт&1&1 \\
			\hline
			5&гигиена&иена&3&3 \\
			\hline
            6&нисан&автоваз&6&6 \\
			\hline
			7&спасибо&пожалуйста&9&9 \\
			\hline
            8&что&кто&1&1 \\
			\hline
            9&ты&тыква&3&3 \\
			\hline
            10&есть&кушать&4&4 \\
			\hline
			11&abba&baab&3&2 \\
			\hline
            12&abcba&bacab&4&2 \\
			\hline
		\end{tabular}
        \end{threeparttable}
	\end{center}
\end{table}

\section{Вывод}

Были представлены всех алгоритмов нахождения расстояния Левенштейна и Дамерау-Левенштейна, которые были описаны в предыдущем разделе. Также в данном разделе была приведена информации о выбранных средствах для разработки алгоритмов.
