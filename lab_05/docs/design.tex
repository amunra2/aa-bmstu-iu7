\chapter{Конструкторская часть}
В этом разделе будут представлено описание используемых типов данных, а также схемы алгоритмов конвейрной и линейной обработки матриц.

\section{Описание используемых типов данных}

При реализации алгоритмов будут использованы следующие типы данных:

\begin{itemize}
	\item количество задач (матриц) - целое число типа \textit{int};
	\item размер матрицы - целое число типа \textit{int};
	\item структура \textit{matrix\_s} - содержит информацию о матрице -- сама матрицы (имеет тип std::vector<std::vector<double>>), ее размер (size\_t), а также хранит информацию о найденном среднем арифметическом и максимальном элементе;
	\item структура \textit{queues\_t} - содержит информацию об очередях (описывает их, каждая очредь имеет тип данных std::queue<matrix\_t>).
\end{itemize}


\section{Структура разрабатываемого ПО}

В данном ПО будет реализован метод структурного программирования. Для взаимодействия с пользователем будет разработано меню, которое будет предоставлять возможность выбрать нужный алгоритм - линейный или конвейерный, замерить время и вывести информацию об этапах. 

Для работы будут разработаны следующие процедуры:

\begin{itemize}
	\item процедура, реализующая генерацию квадратной матрицы по ее размеру, входные данные - размер матрицы, выходные - полученная матрица;
	\item процедура вывода полученной после обработки матрицы на экран (для отладки), входные данные - матрица, выходные - выведенная н экран матрица;
	\item процедуры обработки матрицы (нахождение среднего арифметического, макимального элемента, а также вставка их в матрицу), входные данные - матрица (и среднее арифметическое и максимальный элемент - для заполнения матрица), выходные - среднее арифметическое (задача 1), максимальный элемент (задача 2), измененная матрица (задача 3);
	\item процедуры, реализующие линейную и конвейерную обработки этапов матрицы, входные данные - кол-во задач (матриц), размер матриц, выходные - лог работы программы с выводом времени; 
	\item процедуры замера времени поэтапной орбаботки матрицы с использованием конвейера и линейного подхода, входные данные - количество матриц, начальный размер матрицы, конечный размер матрицы (если время замеряется для разных размеров матриц), размер матриц, начальное количество матриц, конечное количество матрицы (если время замеряется для разного количества задач (матриц)), выходные - результаты замеров времени;
	\item процедура для построения графиков по полученным временным замерам, входные данные - замеры времени, выходные - график.
\end{itemize}


\section{Схемы алгоритмов}
На рисунке \ref{img:linear} представлена схема алгоритма линейной обработки матрицы. На рисунках \ref{img:main_thread} схема алгоритма конвейрной обработки матрицы, а на рисунках \ref{img:thread1}-\ref{img:thread3} - схемы потоков обработки матрицы (ленты конвейера).

\imgScale{0.4}{linear}{Схема алгоритма линейной обработки матрицы}
\imgScale{0.6}{main_thread}{Схема конвейрной обработки матрицы}
\imgScale{0.6}{thread1}{Схема 1 потока обработки матрицы - нахождение среднего арифметического}
\imgScale{0.6}{thread2}{Схема 2 потока обработки матрицы - нахождение максимального элемента}
\imgScale{0.6}{thread3}{Схема 3 потока обработки матрицы - заполнения матрицы новыми значенями}

\clearpage


\section{Классы эквивалентности при тестировании}

Для тестирования выделены классы эквивалентности, представленные ниже.

\begin{enumerate}
	\item Неверно выбран пункт меню - не число или число, меньшее 0 или большее 4.
	\item Неверно введено количество матриц - не число или число, меньшее 1.
	\item Неверно введен размер матриц - не число или число, меньшее 2.
	\item Корректный ввод всех параметров.
\end{enumerate}


\section{Вывод}

В данном разделе были построены схемы алгоритмов, рассматриваемых в лабораторной работе, были описаны классы эквивалентности для тестирования, структура программы.
