\chapter*{Введение}
\addcontentsline{toc}{chapter}{Введение}

Использование параллельной обработки открывает новые способы для ускорения работы программ. Конвейрная обработка является одним из примеров, где использование принципов параллельности помогает ускорить обработку данных. Суть та же, что и при работе реальных конвейрных лент - материал (данное) поступает на обработку, после окончания обработки материал передается на место следующего обработчина, при этом предыдыдущий обработчик не ждет полного цикла обработки материала, а получает новый материал и работает с ним.


\textbf{Целью данной работы} является изучение принципов конвейрной обработки данных. 
Для достижения поставленной цели необходимо выполнить следующие задачи:
\begin{itemize}
	\item изучить основы конвейрной обработки данных;
    \item описать алгоритмы обработки матрицы, которые будут использоваться в текущей лабораторной работе;
    \item привести схемы конвейрной и линейной обработок;
    \item описать используемые типы и структуры данных;
    \item описать структуру разрабатываемого программного обеспечения;
    \item реализовать разработанный алгоритм;
    \item провести функциональное тестирование разработанного алгоритма;
    \item провести сравнительный анализ по времени для реализованного алгоритма;
    \item подготовить отчет по лабораторной работе.
\end{itemize}
