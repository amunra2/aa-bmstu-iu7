\chapter{Исследовательская часть}

В данном разделе будут приведены примеры работы программа, а также проведен сравнительный анализ алгоритмов при различных ситуациях на основе полученных данных.

\section{Технические характеристики}

Технические характеристики устройства, на котором выполнялось тестирование представлены далее:

\begin{itemize}
    \item операционная система: Ubuntu 20.04.3 \cite{ubuntu} Linux \cite{linux} x86\_64;
    \item память: 8 GiB;
    \item процессор: Intel® Core™ i5-7300HQ CPU @ 2.50GHz \cite{intel};
    \item 4 физических ядра, 4 логических ядра \cite{intel}.
\end{itemize}

При тестировании ноутбук был включен в сеть электропитания. Во время тестирования ноутбук был нагружен только встроенными приложениями окружения, а также системой тестирования.

\section{Демонстрация работы программы}

На рисунке \ref{img:example_linear} представлен результат работы программы при линейной обработке, а на рисунке \ref{img:example_conveyor} -- при конвейерной.

\imgHeight{200mm}{example_linear}{Пример работы программы (линейная)}
\imgHeight{200mm}{example_conveyor}{Пример работы программы (конвейерная)}
\clearpage

\section{Время выполнения алгоритмов}

Как было сказано выше, используется функция замера процессорного времени \textit{std::chrono::system\_clock::now(...)} из библиотеки $chrono$ на C++. Функция возвращает процессорное время типа float в секундах.

Использовать функцию приходится дважды, затем из конечного времени нужно вычесть начальное, чтобы получить результат.

Замеры проводились для разного рамзера матриц, а также для разного количества матриц, чтобы определить, когда наиболее эффективно использовать конвейерную обработку.

Результаты замеров приведены в таблицах \ref{tbl:time_lin_count}-\ref{tbl:time_conv_size} (время в с).

\begin{table}[h]
    \begin{center}
        \begin{threeparttable}
        \captionsetup{justification=raggedright,singlelinecheck=off}
        \caption{Результаты замеров времени (линейная -- разное кол-во матриц)}
        \label{tbl:time_lin_count}
        \begin{tabular}{|p{6cm}|p{6cm}|}
            \hline
            Кол-во матриц & Время \\
            \hline
            10 & 0.1847 \\ \hline 
            15 & 0.2841 \\ \hline 
            20 & 0.3771 \\ \hline 
            25 & 0.4755 \\ \hline 
            30 & 0.5648 \\ \hline 
            35 & 0.6581 \\ \hline 
            40 & 0.7558 \\ \hline 
            45 & 0.8529 \\ \hline 
            50 & 0.9818 \\ \hline  
		\end{tabular}
    \end{threeparttable}
\end{center}
\end{table}


\begin{table}[h]
    \begin{center}
        \begin{threeparttable}
        \captionsetup{justification=raggedright,singlelinecheck=off}
        \caption{Результаты замеров времени (конвейрная -- разное кол-во матриц)}
        \label{tbl:time_conv_count}
        \begin{tabular}{|p{6cm}|p{6cm}|}
            \hline
            Кол-во матриц & Время \\
            \hline
            10 & 0.1432 \\ \hline 
            15 & 0.1632 \\ \hline 
            20 & 0.2603 \\ \hline 
            25 & 0.3187 \\ \hline 
            30 & 0.3943 \\ \hline 
            35 & 0.3566 \\ \hline 
            40 & 0.4431 \\ \hline 
            45 & 0.5429 \\ \hline 
            50 & 0.6472 \\ \hline 

		\end{tabular}
    \end{threeparttable}
\end{center}
\end{table}


\begin{table}[h]
    \begin{center}
        \begin{threeparttable}
        \captionsetup{justification=raggedright,singlelinecheck=off}
        \caption{Результаты замеров времени (линейная -- разные размеры матриц)}
        \label{tbl:time_lin_size}
        \begin{tabular}{|p{6cm}|p{6cm}|}
            \hline
            Размер матриц & Время \\
            \hline
            1500 & 0.1574 \\ \hline 
            1600 & 0.1797 \\ \hline 
            1700 & 0.201 \\ \hline 
            1800 & 0.2268 \\ \hline 
            1900 & 0.2533 \\ \hline 
            2000 & 0.2818 \\ \hline 
            2100 & 0.3062 \\ \hline 
            2200 & 0.337 \\ \hline 
            2300 & 0.3736 \\ \hline 
            2400 & 0.4045 \\ \hline 
            2500 & 0.4369 \\ \hline 

		\end{tabular}
    \end{threeparttable}
\end{center}
\end{table}


\begin{table}[h]
    \begin{center}
        \begin{threeparttable}
        \captionsetup{justification=raggedright,singlelinecheck=off}
        \caption{Результаты замеров времени (конвейерная -- разные размеры матриц)}
        \label{tbl:time_conv_size}
        \begin{tabular}{|p{6cm}|p{6cm}|}
            \hline
            Размер & Время \\
            \hline 
            1500 & 0.0912 \\ \hline 
            1600 & 0.0969 \\ \hline 
            1700 & 0.1069 \\ \hline 
            1800 & 0.1527 \\ \hline 
            1900 & 0.1604 \\ \hline 
            2000 & 0.1876 \\ \hline 
            2100 & 0.1986 \\ \hline 
            2200 & 0.2524 \\ \hline 
            2300 & 0.2858 \\ \hline 
            2400 & 0.3015 \\ \hline 
            2500 & 0.2936 \\ \hline 
		\end{tabular}
    \end{threeparttable}
\end{center}
\end{table}

\clearpage

Также на рисунках \ref{img:graph_count}--\ref{img:graph_size} приведены графические результаты замеров.

\imgHeight{100mm}{graph_count}{Сравнение по времени линейной и конвейерной обработок для разного кол-ва матриц}
\imgHeight{100mm}{graph_size}{Сравнение по времени линейной и конвейерной обработок для разных размеров матриц}

\clearpage


\section{Вывод}

В результате эксперимента было получено, что использование конвейрной обработки лучше линейной реализации при количестве матриц, равном 10, в 1.3 раза, а при количестве матриц, котрое равно 50, уже в 1.5 раза. Следовательно, конвейерная реализация лучше линейной при увеличении количества задач (матриц).

Также при проведении эксперимента было выявлено, что при увеличении размера матриц конвейерная реализация выдает лучшие результаты. Так, при размере матрицы в 1500 конвейерная реализация лучше в 1.4 раза, чем линейная, а при размере матриц, равном 2500, в 1.5 раза. То есть, следует использововть конвейрную реализацию.