\chapter*{Заключение}
\addcontentsline{toc}{chapter}{Заключение}

В результате исследования было определено, что конвейерная реализация лучше линейной при изменении обоих критериев -- количества задач (матриц) и при увеличении размеров матриц. Так, использование конвейрного подхода дает прирост в 1.3 раза при 10 матрицах и 1.5 раза при 50 матрицах, а при размере матриц в 1500 время лучше в 1.4 раза, а при 2500 -- в 1.5 раз.

Цель, которая была поставлена в начале лабораторной работы была достигнута, а также в ходе выполнения лабораторной работы были решены следующие задачи:

\begin{itemize}
	\item были изучены принципы конвейерной разработки;
    \item были реализованы алгоритмы обрбаотки матрицы -- нахождение среднего арифметического, нахождение максимального элемента и заполнение матрицы найденным значениями;
	\item проведен сравнительный анализ по времени алгоритмов конвейрной и линейной обработок на разном количестве задач (матриц);
	\item проведен сравнительный анализ по времени алгоритмов конвейрной и линейной обработок на разном размере матриц;
	\item подготовлен отчет о лабораторной работе.
\end{itemize}