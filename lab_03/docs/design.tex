\chapter{Конструкторская часть}
В данном разделе будут рассмотрены схемы алгоритмов сортировок (вставками, перемешиванием и гномья), а также найдена их трудоемкость

\section{Разработка алгоритмов}
На рисунках 2.1, 2.2 и 2.3 представлены схемы алгоритмов сортировки - ставками, перемешиванием и гномья

\img{170mm}{insert_scheme.png}{Схема алгоритма сортировки вставками}
\img{220mm}{shaker_scheme1.png}{Схема алгоритма сортировки перемешиванием - 1}
\img{220mm}{shaker_scheme2.png}{Схема алгоритма сортировки перемешиванием - 2}
\img{220mm}{gnome_scheme.png}{Схема алгоритма гномьей сортировки}

\clearpage

\section{Модель вычислений для проведения оценки трудоемкости}
Введем модель вычислений, которая потребуется для определния трудоемкости каждого отдельно взятого алгоритма сортировки:

\begin{enumerate}
	\item операции из списка (\ref{for:operations}) имеют трудоемкость равную 1;
	\begin{equation}
		\label{for:operations}
		+, -, /, *, \%, =, +=, -=, *=, /=, \%=, ==, !=, <, >, <=, >=, [], ++, {-}-
	\end{equation}
	\item трудоемкость оператора выбора \code{if условие then A else B} рассчитывается, как (\ref{for:if});
	\begin{equation}
		\label{for:if}
		f_{if} = f_{\text{условия}} +
		\begin{cases}
			f_A, & \text{если условие выполняется,}\\
			f_B, & \text{иначе.}
		\end{cases}
	\end{equation}
	\item трудоемкость цикла рассчитывается, как (\ref{for:cycle});
	\begin{equation}
		\label{for:cycle}
		f_{for} = f_{\text{инициализации}} + f_{\text{сравнения}} + N(f_{\text{тела}} + f_{\text{инкремент}} + f_{\text{сравнения}})
	\end{equation}
	\item трудоемкость вызова функции равна 0.
\end{enumerate}


\section{Трудоёмкость алгоритмов}

\subsection{Алгоритм сортировки перемешиванием}

\begin{itemize}
	\item Трудоёмкость сравнения внешнего цикла $while(swap == True)$, которая равна (\ref{for:outer_cycle}):
	\begin{equation}
		\label{for:outer_cycle}
		f_{outer} = 1 + 2 \cdot (N - 1)
	\end{equation}
	\item Суммарная трудоёмкость внутренних циклов, количество итераций которых меняется в промежутке $[1..N-1]$, которая равна (\ref{for:bubble_inner}):
	\begin{equation}
		\label{for:bubble_inner}
		f_{inner} = 5(N - 1) + \frac{2 \cdot (N - 1)}{2} \cdot (3 + f_{if})
	\end{equation}
	\item Трудоёмкость условия во внутреннем цикле, которая равна (\ref{for:bubble_if}):
	\begin{equation}
		\label{for:bubble_if}
		f_{if} = 4 + \begin{cases}
			0, & \text{л.с.}\\
			9, & \text{х.с.}\\
		\end{cases}
	\end{equation}
\end{itemize}

Трудоёмкость в лучшем случае (\ref{for:shaker_best}):
\begin{equation}
	\label{for:shaker_best}
	f_{best} = -3 + \frac{3}{2} N \approx \frac{3}{2} N = O(N)
\end{equation}

Трудоёмкость в худшем случае (\ref{for:shaker_worst}):
\begin{equation}
	\label{for:shaker_worst}
	f_{worst} = -3 - 8N + 8N^2 \approx 8N^2 = O(N^2)
\end{equation}


\subsection{Алгоритм гномьей сортировки}

Трудоёмкость в лучшем случае (при уже отсортированном массиве) (\ref{for:gnome_best}):
\begin{equation}
	\label{for:gnome_best}
    f_{best} = 1 + N(4 + 1) = 5N + 1 = O(N)
	% f_{best} = -3 + \frac{3}{2} N + \approx \frac{3}{2} N = O(N)
\end{equation}

Трудоёмкость в худшем случае (при массиве, отсортированном в обратном порядке) (\ref{for:gnome_worst}):
\begin{equation}
	\label{for:gnome_worst}
    f_{worst} = 1 + N(4 + (N - 1) * (7 + 1 + 2)) = 10N^2 - 6N + 1 = O(N^2)
	% f_{worst} = -3 - 8N + 8N^2 \approx 8N^2 = O(N^2)
\end{equation}


\subsection{Алгоритм сортировки вставками}
Трудоемкость данного алгоритма может быть рассчитана с использованием той же модели подсчета трудоемкости.

Трудоемкость алгоритма сортировки вставками:
\begin{itemize}
	\item в лучщем случае - $O(N)$
    \item в худшем случае - $O(N^2)$ \newline
\end{itemize}



\section*{Вывод}

Были разработаны схемы всех трех алгоритмов сортировки. Также для каждого из них были рассчитаны и оценены лучшие и худшие случаи.
