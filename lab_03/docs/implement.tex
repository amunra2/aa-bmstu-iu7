\chapter{Технологическая часть}

\section{Требования к ПО}
Ряд требований к программе:
\begin{itemize}
	\item на вход подается массив целых чисел 
    \item отсортированный массив, который был задан в предыдущем пункте \newline
\end{itemize}


\section{Средства реализации}
В данной работе для реализации был выбран язык программирования $Python \cite{python-lang}$. Выбор обсуловлен желанием попрактиковать свои умения работы с даным ЯП. Также $Python$ предоставляет широкий выбор библиотек для комфортной работы.

Время работы было замерено с помощью функции \textit{process\_time(...)} из библиотеки $time \cite{python-lang-time}$.

\section{Сведения о модулях программы}
Программа состоит из двух модулей:
\begin{itemize}
	\item $main.py$ - файл, содержащий весь служебный код;
    \item $sorts.py$ - файл, содержащий код всех сортировок. \newline
\end{itemize}


\section{Листинги кода}

В листингах \ref{lst:insert_sort}, \ref{lst:shaker_sort}, \ref{lst:gnomme_sort} представлены реализации алгоритмов сортировок (перемешиванием, вставками и гномьей).

\begin{center}
\captionsetup{justification=raggedright,singlelinecheck=off}
\begin{lstlisting}[label=lst:insert_sort,caption=Алгоритм сортировки вставками]
    def insertion_sort(arr, n):

    for i in range(1, n):
        j = i - 1
        tmp = arr[i]

        while (j >= 0 and arr[j] > tmp):
            arr[j + 1] = arr[j]
            j -= 1

        arr[j + 1] = tmp

    return arr
\end{lstlisting}
\end{center}


\begin{center}
\captionsetup{justification=raggedright,singlelinecheck=off}
\begin{lstlisting}[label=lst:shaker_sort,caption=Алгоритм сортировки перемешиванием]
    def shaker_sort(arr, n):

    left = 0
    right = n - 1

    swapped = True

    while(swapped):
        swapped = False

        for i in range(left, right):
            if (arr[i] > arr[i + 1]):
                tmp = arr[i]
                arr[i] = arr[i + 1]
                arr[i + 1] = tmp

                swapped = True

        if (swapped == False):
            break

        swapped = False
        right -= 1

        for i in range(right - 1, left - 1, -1):
            if (arr[i] > arr[i + 1]):
                tmp = arr[i]
                arr[i] = arr[i + 1]
                arr[i + 1] = tmp

                swapped = True

        left += 1

    return arr
\end{lstlisting}
\end{center}



\begin{center}
\captionsetup{justification=raggedright,singlelinecheck=off}
\begin{lstlisting}[label=lst:gnomme_sort,caption=Алгоритм гномьей сортировки]
    def gnomme_sort(arr, n):
    
    i = 1

    while (i < n):
        if (arr[i] < arr[i - 1]):
            tmp = arr[i]
            arr[i] = arr[i - 1]
            arr[i - 1] = tmp

            if (i > 1):
                i -= 1
        else:
            i += 1

    return arr
\end{lstlisting}
\end{center}

\section{Функциональные тесты}

В таблице \ref{tbl:functional_test} приведены тесты для функций, реализующих алгоритмы сортировки. Тесты \textit{для всех сортировок} пройдены успешно.


\begin{table}[h]
	\begin{center}
		\begin{threeparttable}
		\captionsetup{justification=raggedleft,singlelinecheck=off}
		\caption{\label{tbl:functional_test} Функциональные тесты}
		\begin{tabular}{|c|c|c|}
			\hline
			Входной массив & Ожидаемый результат & Результат \\ 
			\hline
			$[1, 2, 3, 4, 5]$ & $[1, 2, 3, 4, 5]$  & $[1, 2, 3, 4, 5]$\\
			$[5, 4, 3, 2, 1]$  & $[1, 2, 3, 4, 5]$ & $[1, 2, 3, 4, 5]$\\
			$[9, 7, -5, 1, 4]$  & $[-5, 1, 4, 7, 9]$  & $[-5, 1, 4, 7, 9]$\\
			$[5]$  & $[5]$  & $[5]$\\
			$[]$  & $[]$  & $[]$\\
			\hline
		\end{tabular}
    \end{threeparttable}
	\end{center}
\end{table}


\section*{Вывод}

Были разработаны схемы всех трех алгоритмов сортировки. Для каждого алгоритма была вычислена трудоемкость и оценены лучший и худший случаи.
