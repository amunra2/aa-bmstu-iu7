\chapter*{Введение}
\addcontentsline{toc}{chapter}{Введение}

Сортировка - перегруппировка некой последовательности, или кортежа, в определенном порядке. Это одна из главных процудур обработки структурированных данных. Расположение элементов в определенном порядке позволяет более эффективно проводить работу с последовательностью данных, в частности при поиске некоторых данных. \newline

Существует множество алгоритмов сортировки, но любой алгоритм сортировки имеет:
\begin{itemize}
	\item сравнение, которое определяет, как упорядочена пара элементов;
    \item перестановка для смены элементов местами;
    \item алгоритм сортировки, использующий сравнение и перестановки. \newline
\end{itemize}

Что касается самого поиска, то при работе с отсортированным набором данных время, которое нужно на нахождение элемента, пропорционально логарифму количства элементов. Последовательность, данные которой расположены в хаотичном порядке, занимает время, которое пропорционально количству элементов, что куда больше логарифма. \newline

\textbf{Цель работы:} изучение и исследование трудоемкости алгоритмов сортировки. \newline

\textbf{Задачи работы.}
\begin{enumerate}
	\item Изучить и реализовать алгоритмы сортировки: слияние, вставки, шейкер.
    \item Рассмотреть существующие решения.
    \item Разработать алгоритмы сортировок.
    \item Провести сравнительный анализ трудоемкости алгоритмов на основе теоретических расчетов.
    \item Провести сравнительный анализ реализаций алгоритмов по затраченному процессорному времени и памяти.
	\item Описать и обосновать полученные результаты в отчете о выполненной лабораторной работе, выполненного как расчётно-пояснительная записка к работе.
\end{enumerate}