\chapter{Аналитическая часть}
В этом разделе будут рассмотрены алгоритмы сортировок - вставками, перемешиванием, гномья.

\section{Сортировка вставками}
\textbf{Сортировка вставками \cite{insertion-sort}} - размещение элемента входной последовательности на подходящее место выходной последовательности.

Набор данных условно разделяется на входную последовательность и выходную. В начале отсортированная часть пуста. Каждый $i$-ый элемент, начиная с $i = 2$, входной последовательности размещается в уже отсортированную часть до тех пор, пока изначальные данные не будут исчерпаны.


\section{Сортировка перемешиванием}
\textbf{Сортировка перемешиванием \cite{sheyker-sort}} - сортировка, которая является модификацией сортировки пузырьком. Различие состоит в том, что в рамках одной итерации происходит проход по массиву в обоих направлениях. В сортировке пузырьком просходит только проход слева-направо, то е6сть в одном направлении.

Суть сортировки - сначала идет обычный проход слева-направо, как при обычном пузырьке. Затем, начиная с элемента, который находится перед последним отсортированным, начинается проход в обратном направлении. Здесь тикже сравниваются элементы меняются местами при необходимости.


\section{Гномья сортировка}
\textbf{Гномья сортировка \cite{gnomme-sort}} - алгоритм сортировки, который использует только один цикл, что является редкостью.

В этой сортировке массив просматривается селва-направо, при этом сравниваются и, если нужно, меняются соседние элементы. Если происходит обмен элементов, то происходит возвращение на один шаг назад. Если обмена не было - агоритм продолжает просмотр массива в поисках неупорядоченных пар.

\section*{Вывод}
В данной работе необходимо реализовать алгоритмы сортировки, описанные в данном разделе, а также провести их теоритическую оценку и проверить ее экспериментально.