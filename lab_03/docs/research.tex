\chapter{Исследовательская часть}

В данном разеделе будут приедены примеры работы программа, а также проведен сравнительный анализ адгоритмов при различных ситуациях на основе полученных данных.

\section{Технические характеристики}

Технические характеристики устройства, на котором выполнялось тестирование представлены далее:

\begin{itemize}
    \item операционная система: Ubuntu 20.04.3 \cite{ubuntu} Linux \cite{linux} x86\_64;
    \item память: 8 GiB;
    \item процессор: Intel® Core™ i5-7300HQ CPU @ 2.50GHz \cite{intel}.
\end{itemize}

При тестировании ноутбук был включен в сеть электропитания. Во время тестирования ноутбук был нагружен только встроенными приложениями окружения, а также системой тестирования.

\section{Демонстрация работы программы}

На рисунке \ref{img:example} представлен результат работы программы.

\img{120mm}{example}{Пример работы программы}
\clearpage

\section{Время выполнения алгоритмов}

Как было сказано выше, используется функция замера процессорного времени process\_time(...) из библиотеки time на Python. Функция возвращает пользовательское процессорное временя типа float.

Использовать функцию приходится дважды, затем из конечного времени нужно вычесть начальное, чтобы получить результат.

Результаты замеров времени работы алгоритмов сортировки на различных входных данных (в мс) приведены в таблицах \ref{tbl:best}, \ref{tbl:worth} и \ref{tbl:random}.

\begin{table}[h]
	\begin{center}
		\begin{threeparttable}
		\captionsetup{justification=raggedleft,singlelinecheck=off}
		\caption{Отсортированные данные}
		\label{tbl:best}
		\begin{tabular}{|c|c|c|c|}
			\hline
			Размер & Шейкером &  Вставками &  Гномья \\
			\hline
			100 & 0.0205 & 0.0346 & 0.0263 \\ 
			\hline
			200 & 0.0368 & 0.0665 & 0.0507 \\ 
			\hline
			300 & 0.0556 & 0.0648 & 0.0575 \\ 
			\hline
			400 & 0.0474 & 0.0725 & 0.0546 \\ 
			\hline
			500 & 0.0507 & 0.0931 & 0.0681 \\ 
			\hline
			600 & 0.0600 & 0.1121 & 0.0820 \\ 
			\hline
			700 & 0.0695 & 0.1305 & 0.0956 \\ 
			\hline
			800 & 0.0801 & 0.1497 & 0.1148 \\ 
			\hline
			900 & 0.0893 & 0.1671 & 0.1238 \\ 
			\hline
			1000 & 0.1018 & 0.1844 & 0.1391 \\ 
			\hline
		\end{tabular}
		\end{threeparttable}
    \end{center}
\end{table}


\begin{table}[h]
	\begin{center}
		\begin{threeparttable}
		\captionsetup{justification=raggedleft,singlelinecheck=off}
		\caption{Отсортированные в обратном порядке данные}
		\label{tbl:worth}
		\begin{tabular}{|c|c|c|c|}
			\hline
			 Размер & Шейкером &  Вставками &  Гномья \\
			\hline
			100 & 2.1406 & 0.7426 & 2.1195 \\ 
			\hline
			200 & 4.3674 & 2.9388 & 7.7834 \\ 
			\hline
			300 & 9.8322 & 6.6206 & 17.5885 \\ 
			\hline
			400 & 18.0855 & 11.9534 & 31.8696 \\ 
			\hline
			500 & 29.2288 & 18.9838 & 50.5128 \\ 
			\hline
			600 & 43.0116 & 27.6467 & 73.6537 \\ 
			\hline
			700 & 60.0487 & 37.8276 & 101.4856 \\ 
			\hline
			800 & 78.2646 & 49.7082 & 133.4173 \\ 
			\hline
			900 & 99.4545 & 62.8961 & 169.1879 \\ 
			\hline
			1000 & 123.1786 & 77.9326 & 210.1529 \\ 
			\hline
		\end{tabular}
		\end{threeparttable}
    \end{center}
\end{table}


\begin{table}[h]
	\begin{center}
		\begin{threeparttable}
		\captionsetup{justification=raggedleft,singlelinecheck=off}
		\caption{Случайные данные}
		\label{tbl:random}
		\begin{tabular}{|c|c|c|c|}
			\hline
			 Размер & Шейкером &  Вставками &  Гномья \\
			\hline
			100 & 0.9844 & 0.3929 & 1.0544 \\ 
			\hline
			200 & 2.6970 & 1.4714 & 3.8801 \\ 
			\hline
			300 & 6.0501 & 3.3485 & 8.9118 \\ 
			\hline
			400 & 10.6126 & 5.6650 & 15.1574 \\ 
			\hline
			500 & 17.6962 & 9.3086 & 24.8351 \\ 
			\hline
			600 & 25.4620 & 13.6112 & 36.4204 \\ 
			\hline
			700 & 36.9290 & 19.6599 & 52.6627 \\ 
			\hline
			800 & 47.1996 & 25.0265 & 67.0859 \\ 
			\hline
			900 & 60.8530 & 32.3967 & 86.8291 \\ 
			\hline
			1000 & 74.5834 & 40.1647 & 107.6520 \\ 
			\hline
		\end{tabular}
		\end{threeparttable}
    \end{center}
\end{table}


Также на рисунках \ref{img:graph_sorted}, \ref{img:graph_sorted_back}, \ref{img:graph_random} приведены графические результаты замеров работы сортировок в зависимости от размера входного массива.


\img{100mm}{graph_sorted}{Отсортированный массив}
\img{100mm}{graph_sorted_back}{Отсортированный в обратном порядке массив}
\img{100mm}{graph_random}{Случайный массив}
\clearpage

\section*{Вывод}
Исходя из полученных результатов, гномья сортировка при случайно заполненном массиве, а также при обратно отсортированном работает дольше всех (примерно в 1.5 дольше, чем сортировка перемешиванием и в 2 раза дольше сортировки вставками), при этом сортировка методом вставок показала себя лучше всех. Что касается уже отсортированных данных, то лучше всего себя здесь показала сортировка перемешиванием, в то время, как метод вставок оказался худшим.

Теоретические результаты замеров и полученные практически результаты совпадают.